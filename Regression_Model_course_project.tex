\documentclass[]{article}
\usepackage{lmodern}
\usepackage{amssymb,amsmath}
\usepackage{ifxetex,ifluatex}
\usepackage{fixltx2e} % provides \textsubscript
\ifnum 0\ifxetex 1\fi\ifluatex 1\fi=0 % if pdftex
  \usepackage[T1]{fontenc}
  \usepackage[utf8]{inputenc}
\else % if luatex or xelatex
  \ifxetex
    \usepackage{mathspec}
  \else
    \usepackage{fontspec}
  \fi
  \defaultfontfeatures{Ligatures=TeX,Scale=MatchLowercase}
\fi
% use upquote if available, for straight quotes in verbatim environments
\IfFileExists{upquote.sty}{\usepackage{upquote}}{}
% use microtype if available
\IfFileExists{microtype.sty}{%
\usepackage{microtype}
\UseMicrotypeSet[protrusion]{basicmath} % disable protrusion for tt fonts
}{}
\usepackage[margin=1in]{geometry}
\usepackage{hyperref}
\hypersetup{unicode=true,
            pdfborder={0 0 0},
            breaklinks=true}
\urlstyle{same}  % don't use monospace font for urls
\usepackage{color}
\usepackage{fancyvrb}
\newcommand{\VerbBar}{|}
\newcommand{\VERB}{\Verb[commandchars=\\\{\}]}
\DefineVerbatimEnvironment{Highlighting}{Verbatim}{commandchars=\\\{\}}
% Add ',fontsize=\small' for more characters per line
\usepackage{framed}
\definecolor{shadecolor}{RGB}{248,248,248}
\newenvironment{Shaded}{\begin{snugshade}}{\end{snugshade}}
\newcommand{\KeywordTok}[1]{\textcolor[rgb]{0.13,0.29,0.53}{\textbf{#1}}}
\newcommand{\DataTypeTok}[1]{\textcolor[rgb]{0.13,0.29,0.53}{#1}}
\newcommand{\DecValTok}[1]{\textcolor[rgb]{0.00,0.00,0.81}{#1}}
\newcommand{\BaseNTok}[1]{\textcolor[rgb]{0.00,0.00,0.81}{#1}}
\newcommand{\FloatTok}[1]{\textcolor[rgb]{0.00,0.00,0.81}{#1}}
\newcommand{\ConstantTok}[1]{\textcolor[rgb]{0.00,0.00,0.00}{#1}}
\newcommand{\CharTok}[1]{\textcolor[rgb]{0.31,0.60,0.02}{#1}}
\newcommand{\SpecialCharTok}[1]{\textcolor[rgb]{0.00,0.00,0.00}{#1}}
\newcommand{\StringTok}[1]{\textcolor[rgb]{0.31,0.60,0.02}{#1}}
\newcommand{\VerbatimStringTok}[1]{\textcolor[rgb]{0.31,0.60,0.02}{#1}}
\newcommand{\SpecialStringTok}[1]{\textcolor[rgb]{0.31,0.60,0.02}{#1}}
\newcommand{\ImportTok}[1]{#1}
\newcommand{\CommentTok}[1]{\textcolor[rgb]{0.56,0.35,0.01}{\textit{#1}}}
\newcommand{\DocumentationTok}[1]{\textcolor[rgb]{0.56,0.35,0.01}{\textbf{\textit{#1}}}}
\newcommand{\AnnotationTok}[1]{\textcolor[rgb]{0.56,0.35,0.01}{\textbf{\textit{#1}}}}
\newcommand{\CommentVarTok}[1]{\textcolor[rgb]{0.56,0.35,0.01}{\textbf{\textit{#1}}}}
\newcommand{\OtherTok}[1]{\textcolor[rgb]{0.56,0.35,0.01}{#1}}
\newcommand{\FunctionTok}[1]{\textcolor[rgb]{0.00,0.00,0.00}{#1}}
\newcommand{\VariableTok}[1]{\textcolor[rgb]{0.00,0.00,0.00}{#1}}
\newcommand{\ControlFlowTok}[1]{\textcolor[rgb]{0.13,0.29,0.53}{\textbf{#1}}}
\newcommand{\OperatorTok}[1]{\textcolor[rgb]{0.81,0.36,0.00}{\textbf{#1}}}
\newcommand{\BuiltInTok}[1]{#1}
\newcommand{\ExtensionTok}[1]{#1}
\newcommand{\PreprocessorTok}[1]{\textcolor[rgb]{0.56,0.35,0.01}{\textit{#1}}}
\newcommand{\AttributeTok}[1]{\textcolor[rgb]{0.77,0.63,0.00}{#1}}
\newcommand{\RegionMarkerTok}[1]{#1}
\newcommand{\InformationTok}[1]{\textcolor[rgb]{0.56,0.35,0.01}{\textbf{\textit{#1}}}}
\newcommand{\WarningTok}[1]{\textcolor[rgb]{0.56,0.35,0.01}{\textbf{\textit{#1}}}}
\newcommand{\AlertTok}[1]{\textcolor[rgb]{0.94,0.16,0.16}{#1}}
\newcommand{\ErrorTok}[1]{\textcolor[rgb]{0.64,0.00,0.00}{\textbf{#1}}}
\newcommand{\NormalTok}[1]{#1}
\usepackage{graphicx,grffile}
\makeatletter
\def\maxwidth{\ifdim\Gin@nat@width>\linewidth\linewidth\else\Gin@nat@width\fi}
\def\maxheight{\ifdim\Gin@nat@height>\textheight\textheight\else\Gin@nat@height\fi}
\makeatother
% Scale images if necessary, so that they will not overflow the page
% margins by default, and it is still possible to overwrite the defaults
% using explicit options in \includegraphics[width, height, ...]{}
\setkeys{Gin}{width=\maxwidth,height=\maxheight,keepaspectratio}
\IfFileExists{parskip.sty}{%
\usepackage{parskip}
}{% else
\setlength{\parindent}{0pt}
\setlength{\parskip}{6pt plus 2pt minus 1pt}
}
\setlength{\emergencystretch}{3em}  % prevent overfull lines
\providecommand{\tightlist}{%
  \setlength{\itemsep}{0pt}\setlength{\parskip}{0pt}}
\setcounter{secnumdepth}{0}
% Redefines (sub)paragraphs to behave more like sections
\ifx\paragraph\undefined\else
\let\oldparagraph\paragraph
\renewcommand{\paragraph}[1]{\oldparagraph{#1}\mbox{}}
\fi
\ifx\subparagraph\undefined\else
\let\oldsubparagraph\subparagraph
\renewcommand{\subparagraph}[1]{\oldsubparagraph{#1}\mbox{}}
\fi

%%% Use protect on footnotes to avoid problems with footnotes in titles
\let\rmarkdownfootnote\footnote%
\def\footnote{\protect\rmarkdownfootnote}

%%% Change title format to be more compact
\usepackage{titling}

% Create subtitle command for use in maketitle
\newcommand{\subtitle}[1]{
  \posttitle{
    \begin{center}\large#1\end{center}
    }
}

\setlength{\droptitle}{-2em}

  \title{}
    \pretitle{\vspace{\droptitle}}
  \posttitle{}
    \author{}
    \preauthor{}\postauthor{}
    \date{}
    \predate{}\postdate{}
  

\begin{document}

\begin{Shaded}
\begin{Highlighting}[]
\KeywordTok{library}\NormalTok{(ggplot2)}
\KeywordTok{data}\NormalTok{(mtcars)}
\NormalTok{mtcars}\OperatorTok{$}\NormalTok{cyl  <-}\StringTok{ }\KeywordTok{factor}\NormalTok{(mtcars}\OperatorTok{$}\NormalTok{cyl)}
\NormalTok{mtcars}\OperatorTok{$}\NormalTok{vs   <-}\StringTok{ }\KeywordTok{factor}\NormalTok{(mtcars}\OperatorTok{$}\NormalTok{vs)}
\NormalTok{mtcars}\OperatorTok{$}\NormalTok{gear <-}\StringTok{ }\KeywordTok{factor}\NormalTok{(mtcars}\OperatorTok{$}\NormalTok{gear)}
\NormalTok{mtcars}\OperatorTok{$}\NormalTok{carb <-}\StringTok{ }\KeywordTok{factor}\NormalTok{(mtcars}\OperatorTok{$}\NormalTok{carb)}
\NormalTok{mtcars}\OperatorTok{$}\NormalTok{am   <-}\StringTok{ }\KeywordTok{factor}\NormalTok{(mtcars}\OperatorTok{$}\NormalTok{am,}\DataTypeTok{labels=}\KeywordTok{c}\NormalTok{(}\StringTok{"Automatic"}\NormalTok{,}\StringTok{"Manual"}\NormalTok{))}
\CommentTok{#automatic is better for MPG, but we will now quantify his difference}
\KeywordTok{aggregate}\NormalTok{(mpg}\OperatorTok{~}\NormalTok{am, }\DataTypeTok{data =}\NormalTok{ mtcars, mean)}
\end{Highlighting}
\end{Shaded}

\begin{verbatim}
##          am      mpg
## 1 Automatic 17.14737
## 2    Manual 24.39231
\end{verbatim}

\begin{Shaded}
\begin{Highlighting}[]
\CommentTok{#         am      mpg}
\CommentTok{#1 Automatic 17.14737}
\CommentTok{#2    Manual 24.39231}
\CommentTok{#automatic cars have an MPG 7.25 lower than manual cars}
\NormalTok{D_automatic <-}\StringTok{ }\NormalTok{mtcars[mtcars}\OperatorTok{$}\NormalTok{am }\OperatorTok{==}\StringTok{ "Automatic"}\NormalTok{,]}
\NormalTok{D_manual <-}\StringTok{ }\NormalTok{mtcars[mtcars}\OperatorTok{$}\NormalTok{am }\OperatorTok{==}\StringTok{ "Manual"}\NormalTok{,]}
\KeywordTok{t.test}\NormalTok{(D_automatic}\OperatorTok{$}\NormalTok{mpg, D_manual}\OperatorTok{$}\NormalTok{mpg)}
\end{Highlighting}
\end{Shaded}

\begin{verbatim}
## 
##  Welch Two Sample t-test
## 
## data:  D_automatic$mpg and D_manual$mpg
## t = -3.7671, df = 18.332, p-value = 0.001374
## alternative hypothesis: true difference in means is not equal to 0
## 95 percent confidence interval:
##  -11.280194  -3.209684
## sample estimates:
## mean of x mean of y 
##  17.14737  24.39231
\end{verbatim}

\begin{Shaded}
\begin{Highlighting}[]
\CommentTok{#}
\CommentTok{#   Welch Two Sample t-test}
\CommentTok{#}
\CommentTok{#data:  D_automatic$mpg and D_manual$mpg}
\CommentTok{#t = -3.7671, df = 18.332, p-value = 0.001374}
\CommentTok{#alternative hypothesis: true difference in means is not equal to 0}
\CommentTok{#95 percent confidence interval:}
\CommentTok{# -11.280194  -3.209684}
\CommentTok{#sample estimates:}
\CommentTok{#mean of x mean of y }
\CommentTok{# 17.14737  24.39231 }
\CommentTok{#this is a significant difference}
\NormalTok{init <-}\StringTok{ }\KeywordTok{lm}\NormalTok{(mpg }\OperatorTok{~}\StringTok{ }\NormalTok{am, }\DataTypeTok{data =}\NormalTok{ mtcars)}
\KeywordTok{summary}\NormalTok{(init)}
\end{Highlighting}
\end{Shaded}

\begin{verbatim}
## 
## Call:
## lm(formula = mpg ~ am, data = mtcars)
## 
## Residuals:
##     Min      1Q  Median      3Q     Max 
## -9.3923 -3.0923 -0.2974  3.2439  9.5077 
## 
## Coefficients:
##             Estimate Std. Error t value Pr(>|t|)    
## (Intercept)   17.147      1.125  15.247 1.13e-15 ***
## amManual       7.245      1.764   4.106 0.000285 ***
## ---
## Signif. codes:  0 '***' 0.001 '**' 0.01 '*' 0.05 '.' 0.1 ' ' 1
## 
## Residual standard error: 4.902 on 30 degrees of freedom
## Multiple R-squared:  0.3598, Adjusted R-squared:  0.3385 
## F-statistic: 16.86 on 1 and 30 DF,  p-value: 0.000285
\end{verbatim}

\begin{Shaded}
\begin{Highlighting}[]
\CommentTok{#}
\CommentTok{#Call:}
\CommentTok{#lm(formula = mpg ~ am, data = mtcars)}
\CommentTok{#}
\CommentTok{#Residuals:}
\CommentTok{#    Min      1Q  Median      3Q     Max }
\CommentTok{#-9.3923 -3.0923 -0.2974  3.2439  9.5077 }
\CommentTok{#}
\CommentTok{#Coefficients:}
\CommentTok{#            Estimate Std. Error t value Pr(>|t|)    }
\CommentTok{#(Intercept)   17.147      1.125  15.247 1.13e-15 ***}
\CommentTok{#amManual       7.245      1.764   4.106 0.000285 ***}
\CommentTok{#---}
\CommentTok{#Signif. codes:  0 ‘***’ 0.001 ‘**’ 0.01 ‘*’ 0.05 ‘.’ 0.1 ‘ ’ 1}
\CommentTok{#}
\CommentTok{#Residual standard error: 4.902 on 30 degrees of freedom}
\CommentTok{#Multiple R-squared:  0.3598,   Adjusted R-squared:  0.3385 }
\CommentTok{#F-statistic: 16.86 on 1 and 30 DF,  p-value: 0.000285}
\CommentTok{#build a multivariate linear regression}
\NormalTok{betterFit <-}\StringTok{ }\KeywordTok{lm}\NormalTok{(mpg}\OperatorTok{~}\NormalTok{am }\OperatorTok{+}\StringTok{ }\NormalTok{cyl }\OperatorTok{+}\StringTok{ }\NormalTok{disp }\OperatorTok{+}\StringTok{ }\NormalTok{hp }\OperatorTok{+}\StringTok{ }\NormalTok{wt, }\DataTypeTok{data =}\NormalTok{ mtcars)}
\KeywordTok{anova}\NormalTok{(init, betterFit)}
\end{Highlighting}
\end{Shaded}

\begin{verbatim}
## Analysis of Variance Table
## 
## Model 1: mpg ~ am
## Model 2: mpg ~ am + cyl + disp + hp + wt
##   Res.Df    RSS Df Sum of Sq      F    Pr(>F)    
## 1     30 720.90                                  
## 2     25 150.41  5    570.49 18.965 8.637e-08 ***
## ---
## Signif. codes:  0 '***' 0.001 '**' 0.01 '*' 0.05 '.' 0.1 ' ' 1
\end{verbatim}

\begin{Shaded}
\begin{Highlighting}[]
\CommentTok{#Analysis of Variance Table}
\CommentTok{#}
\CommentTok{#Model 1: mpg ~ am}
\CommentTok{#Model 2: mpg ~ am + cyl + disp + hp + wt}
\CommentTok{#  Res.Df    RSS Df Sum of Sq      F    Pr(>F)    }
\CommentTok{#1     30 720.90                                  }
\CommentTok{#2     25 150.41  5    570.49 18.965 8.637e-08 ***}
\CommentTok{#---}
\CommentTok{#Signif. codes:  0 ‘***’ 0.001 ‘**’ 0.01 ‘*’ 0.05 ‘.’ 0.1 ‘ ’ 1}
\CommentTok{#double-check the residuals for non-normality (Appendix - Plot 3) and can see they are all normally distributed and homoskedastic}
\KeywordTok{summary}\NormalTok{(betterFit)}
\end{Highlighting}
\end{Shaded}

\begin{verbatim}
## 
## Call:
## lm(formula = mpg ~ am + cyl + disp + hp + wt, data = mtcars)
## 
## Residuals:
##     Min      1Q  Median      3Q     Max 
## -3.9374 -1.3347 -0.3903  1.1910  5.0757 
## 
## Coefficients:
##              Estimate Std. Error t value Pr(>|t|)    
## (Intercept) 33.864276   2.695416  12.564 2.67e-12 ***
## amManual     1.806099   1.421079   1.271   0.2155    
## cyl6        -3.136067   1.469090  -2.135   0.0428 *  
## cyl8        -2.717781   2.898149  -0.938   0.3573    
## disp         0.004088   0.012767   0.320   0.7515    
## hp          -0.032480   0.013983  -2.323   0.0286 *  
## wt          -2.738695   1.175978  -2.329   0.0282 *  
## ---
## Signif. codes:  0 '***' 0.001 '**' 0.01 '*' 0.05 '.' 0.1 ' ' 1
## 
## Residual standard error: 2.453 on 25 degrees of freedom
## Multiple R-squared:  0.8664, Adjusted R-squared:  0.8344 
## F-statistic: 27.03 on 6 and 25 DF,  p-value: 8.861e-10
\end{verbatim}

\begin{Shaded}
\begin{Highlighting}[]
\CommentTok{#}
\CommentTok{#Call:}
\CommentTok{#lm(formula = mpg ~ am + cyl + disp + hp + wt, data = mtcars)}
\CommentTok{#}
\CommentTok{#Residuals:}
\CommentTok{#    Min      1Q  Median      3Q     Max }
\CommentTok{#-3.9374 -1.3347 -0.3903  1.1910  5.0757 }
\CommentTok{#}
\CommentTok{#Coefficients:}
\CommentTok{#             Estimate Std. Error t value Pr(>|t|)    }
\CommentTok{#(Intercept) 33.864276   2.695416  12.564 2.67e-12 ***}
\CommentTok{#amManual     1.806099   1.421079   1.271   0.2155    }
\CommentTok{#cyl6        -3.136067   1.469090  -2.135   0.0428 *  }
\CommentTok{#cyl8        -2.717781   2.898149  -0.938   0.3573    }
\CommentTok{#disp         0.004088   0.012767   0.320   0.7515    }
\CommentTok{#hp          -0.032480   0.013983  -2.323   0.0286 *  }
\CommentTok{#wt          -2.738695   1.175978  -2.329   0.0282 *  }
\CommentTok{#---}
\CommentTok{#Signif. codes:  0 ‘***’ 0.001 ‘**’ 0.01 ‘*’ 0.05 ‘.’ 0.1 ‘ ’ 1}
\CommentTok{#}
\CommentTok{#Residual standard error: 2.453 on 25 degrees of freedom}
\CommentTok{#Multiple R-squared:  0.8664,   Adjusted R-squared:  0.8344 }
\CommentTok{#F-statistic: 27.03 on 6 and 25 DF,  p-value: 8.861e-10}
\CommentTok{#}
\KeywordTok{boxplot}\NormalTok{(mpg }\OperatorTok{~}\StringTok{ }\NormalTok{am, }\DataTypeTok{data =}\NormalTok{ mtcars, }\DataTypeTok{col =}\NormalTok{ (}\KeywordTok{c}\NormalTok{(}\StringTok{"red"}\NormalTok{,}\StringTok{"blue"}\NormalTok{)), }\DataTypeTok{ylab =} \StringTok{"Miles Per Gallon"}\NormalTok{, }\DataTypeTok{xlab =} \StringTok{"Transmission Type"}\NormalTok{) }\CommentTok{#Plot 1: Boxplot of MPG by transmission type}
\end{Highlighting}
\end{Shaded}

\includegraphics{Regression_Model_course_project_files/figure-latex/unnamed-chunk-1-1.pdf}

\begin{Shaded}
\begin{Highlighting}[]
\KeywordTok{pairs}\NormalTok{(mpg }\OperatorTok{~}\StringTok{ }\NormalTok{., }\DataTypeTok{data =}\NormalTok{ mtcars) }\CommentTok{#Plot 2: Pairs plot for the data set}
\end{Highlighting}
\end{Shaded}

\includegraphics{Regression_Model_course_project_files/figure-latex/unnamed-chunk-1-2.pdf}

\begin{Shaded}
\begin{Highlighting}[]
\KeywordTok{par}\NormalTok{(}\DataTypeTok{mfrow =} \KeywordTok{c}\NormalTok{(}\DecValTok{2}\NormalTok{,}\DecValTok{2}\NormalTok{)) }\CommentTok{#Plot 3: Check residuals}
\KeywordTok{plot}\NormalTok{(betterFit)}
\end{Highlighting}
\end{Shaded}

\includegraphics{Regression_Model_course_project_files/figure-latex/unnamed-chunk-1-3.pdf}


\end{document}
